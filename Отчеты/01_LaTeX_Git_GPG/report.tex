\documentclass[12pt,a4paper]{article}
\usepackage[utf8x]{inputenc}
\usepackage[russian]{babel}
\usepackage[OT1]{fontenc}
\usepackage{amsmath}
\usepackage{amsfonts}
\usepackage{amssymb}
\usepackage{graphicx}
\usepackage{indentfirst}
\usepackage{fancyvrb}
\setlength{\parskip}{1em}
\setlength{\parindent}{0.5cm}
\author{Федор Кусов}
\title{Отчет по лабораторной работе 1: \Latex{} Git GPG}
\begin{document}
\section{Cистема верстки \TeX{} и расширения \LaTeX{}}
\subsection{Цель работы}
Изучение принципов верстки TEX, создание первого отчета
\subsection{Ход работы}
\subsubsection{Создание минимального файла .tex в простом текстовом редакторе – преамбула, тело документа}


Документ LaTeX — это текстовый файл, содержащий специальные команды языка разметки. Документ делится на преамбулу и тело.

Преамбула содержит информацию про класс документа, использованные пакеты макросов, определения макросов, автора, дату создания документа и другую информацию. Ниже представлена простая преамбула для статьи
\begin{Verbatim}[xleftmargin=.5in,fontsize=\small]
\documentclass[12pt]{article} % Класс документа article, величина шрифта 12пт.
\usepackage[russian]{babel} % Пакет поддержки русского языка
\title{Отчет по лабораторной работе 1: \Latex{} Git GPG} % Заглавие документа
\date{\today} % Дата создания
\end{Verbatim}

Тело документа содержит текст документа и команды разметки. Оно должно находиться между командами $\backslash$begin\{document\} и $\backslash$end\{document\}.
\subsubsection{Компиляция в командной строке – latex, xdvi, pdflatex}
Работа с файлами на языке latex осуществляется при помощи пакета утилит. Вот некоторые из них:
\begin{itemize}
\item latex -- принимает файл на языке \LaTeX{} на вход, выдавая бинарный файл в формате dvi (от DeVice independent -- независимый от устройства). Пример использования:
\begin{Verbatim}[xleftmargin=.5in,fontsize=\small]
latex report.tex
\end{Verbatim}

Результатом работы команды будет файл report.dvi
\item xdvi -- позволяет вывести содержимое .dvi на экран
\begin{Verbatim}[xleftmargin=.5in,fontsize=\small]
xdvi report.dvi
\end{Verbatim}
Выведет report.dvi на экран в графическом виде.
\item pdflatex -- преобразует .tex в формат PDF.
\begin{Verbatim}[xleftmargin=.5in,fontsize=\small]
pdflatex report.tex
\end{Verbatim}
Преобразует report.tex в report.pdf
\end{itemize}
\subsubsection{Оболочка TexMaker, Быстрый старт, Быстрая сборка}
\subsubsection{Создание титульного листа, нескольких разделов, списка, несложной формулы}
\subsubsection{Понятие классов документов, подключаемых пакетов}
\subsubsection{Верстка более сложных формул}
\end{document}